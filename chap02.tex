\section{Extremal function}
\begin{ntn}
Let $M$ be a matrix. We denote $|M|$ the weight of $M$, the number of one-entries in $M$.
\end{ntn}
Usually $|M|$ stands for a determinant of matrix $M$. However, in this paper we do not work with determinants at all so the notation should not lead to misunderstanding.
\begin{defn}
For a matrix $P$ we define $Ex(P,m,n):=\max\{|M||M\in\Mat,\ \PnsmM\}$. We denote $Ex(P,n):=Ex(P,n,n)$.
\end{defn}
\begin{defn}
For a matrix $P$ we define $Ex_{\preceq}(P,m,n):=max\{|M||M\in\Mat,\ \PnimM\}$. We denote $Ex_{\preceq}(P,n):=Ex_{\preceq}(P,n,n)$.
\end{defn}
\begin{obs}
For all $P,m,n$; $Ex_{\preceq}(P,m,n)\leq Ex(P,m,n)$.
\end{obs}
\begin{obs}
If $P\in\{0,1\}^{k\times l}$ has a one-entry at position $[a,b]$, then $$Ex(P,m,n)\geq\Big\{\begin{array}{ll}
m\cdot n & k>m\vee l>m \\
(k-1)n+(l-1)m-(k-1)(l-1) & \text{otherwise.}
\end{array}$$
\end{obs}
\begin{obs}
The same holds for $Ex_{\preceq}(P,m,n).$
\end{obs}
\begin{defn}
$P\in\{0,1\}^{k\times l}$ is \emph{(strongly) minimalist} if
$$Ex(P,m,n)=\Big\{\begin{array}{ll}
m\cdot n & k>m\vee l>m \\
(k-1)n+(l-1)m-(k-1)(l-1) & \text{otherwise.}
\end{array}$$
\end{defn}
\begin{defn}
$P\in\{0,1\}^{k\times l}$ is \emph{weakly minimalist} if
$$Ex_{\preceq}(P,m,n)=\Big\{\begin{array}{ll}
m\cdot n & k>m\vee l>m \\
(k-1)n+(l-1)m-(k-1)(l-1) & \text{otherwise.}
\end{array}$$
\end{defn}
\begin{obs}
If $P$ is strongly minimalist, then $P$ is weakly minimalist.
\end{obs}
\subsection{Known results}
\begin{fct}
\begin{enumerate}
\item $\smm{1}$ is strongly minimalist.
\item If $P\in\Pat$ is strongly minimalist and there is a one-entry in the last row in the $c$-th column, then $P'\in\{0,1\}^{k+1\times l}$, which is created from $P$ by adding a new row having a one-entry only in the $c$-th column, is strongly minimalist.
\item If $P$ is strongly minimalist, then after changing a one-entry into a zero-entry it is still strongly minimalist.
\end{enumerate}
\end{fct}
% TODO: cite properly Interval minors of complete bipartite graphs - Bojan Mohar, Arash Rafiey
% Maybe mention that they were working with a different approach (ordered bigraphs) and also a different definition of an interval minor (allowing transposition) but it is not a problem for us and we show the proof in our notation.
\begin{fct}
Let $P=\smm{1\dots1\\1\dots1}$ have $l$ columns, then $P$ is weakly minimalist.
\end{fct}
\begin{proof}
Let $M\in\Mat$ be a matrix avoiding $P=\{1\}^{2\times l}$ as an interval minor and $A_i=\{j\in[n]|\text{weight of }M[[i],\{j\}]>0\wedge \text{weight of }M[[i+1,m],\{j\}>0]\}$. Clearly $|A_i|\leq l-1$, otherwise $P\preceq M$. Let $b_j$ denote the number of one-entries in the $j$-th column. Each column~$j$ of $M$ appears in at least $b_j-1$ of sets $A_i$, $0\leq i\leq m-2$. It follows that
$$\text{weight of }M=\sum\limits_{j=0}^nb_j=\sum\limits_{j=0}^n(b_j-1)+n\leq\sum\limits_{i=0}^{m-2}|A_i|+n\leq(l-1)(m-1)+n$$
\end{proof}
This result is indeed very important because it shows that there are matrices like $\smm{11\\11}$, which are weakly minimalist, although it is known they are not strongly minimalist.
\begin{fct}
Let $P=\smm{1\dots1\\1\dots1\\1\dots1}$ have $l$ columns, then $P$ is weakly minimalist.
\end{fct}
\begin{proof}
Let $M\in\Mat$ be a matrix avoiding $P=\{1\}^{3\times l}$ as an interval minor and $A_i=\{j\in[n]|\text{ weight of }M[[i-1],\{j\}]>0\wedge \text{ weight of }M[[i+1,m],\{j\}>0\wedge M[i,j]\text{ one-entry}]\}$. Clearly $|A_i|\leq l-1$, otherwise $P\preceq M$. Let $b_j$ denote the number of one-entries in the $j$-th column. Each column~$j$ of $M$ (for which $b_j\geq2$) appears in exactly $b_j-2$ of sets $A_i$, $1\leq i\leq m-1$. It follows that
$$\text{weight of }M=\sum\limits_{j=0}^nb_j=\sum\limits_{j=0}^n(b_j-2)+2n\leq\sum\limits_{i=1}^{m-2}|A_i|+2n\leq(l-1)(m-2)+2n$$
\end{proof}
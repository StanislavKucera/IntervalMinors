\section{Operations with matrices}
\begin{ntn}
When speaking about a class of matrices, unless stated otherwise, we always expect the class to be closed under minors. Also, all classes discussed are non-trivial. That means that there is at least one matrix of size $2$ by $1$ and at least one matrix of size $1$ by $2$ in each class. Moreover, at least one matrix is non-empty.
\end{ntn}
\begin{defn}
Let $\mathcal{F}$ be any class of forbidden matrices. We denote by $Av(\mathcal{F})$ the set of all matrices that avoid every $F\in\mathcal{F}$ as an interval minor.
\end{defn}
\begin{obs}
Let $\mathcal{T}=Av(\mathcal{F})$ for some $\mathcal{F}$. Then $\mathcal{T}$ is closed under minors.
\end{obs}
\begin{obs}
Let $\mathcal{M}$ be a finite class of matrices. There exists a finite set $\mathcal{F}$ such that $\mathcal{M}=Av(\mathcal{F})$.
\end{obs}
\begin{defn}
For matrices $A\in\Mat$ and $B\in\Pat$ we define their \emph{direct sum} as a matrix $C:=A\nearrow B\in\{0,1\}^{m+k\times n+l}$ such that $D[[k+1,m+k],[n]]=A$, $D[[k],[n+1,n+l]]=B$ and the rest is empty. Symmetrically, we define $D:=A\searrow B\in\{0,1\}^{m+k\times n+l}$ such that $C[[m],[n]]=A$, $C[[m+1,m+k],[n+1,n+l]]=B$ and the rest is empty.
\end{defn}
\begin{thm}
$Av(\smm{1&1\\1&1})=\left(Av(\smm{1&0\\0&0})\searrow Av(\smm{0&1\\1&0})\searrow Av(\smm{0&0\\0&1})\right)\cup$\\
$\cup\left(Av(\smm{0&0\\1&0})\nearrow Av(\smm{1&0\\0&1})\nearrow Av(\smm{0&1\\0&0})\right)$.
\end{thm}
\begin{proof}
If follows from Theorem~\ref{t33} and $Av(\smm{1&1\\1&0})=Av(\smm{1&0\\0&0})\searrow Av(\smm{0&1\\1&0})$.
\end{proof}
\begin{ntn}
Let $\mathcal{M}$ be a class of matrices. Denote by $Cl(\mathcal{M})$ a set containing each $M\in\mathcal{M}$ closed under direct sum and minors.
\end{ntn}\begin{defn}
Let $M\in\Mat$ be a matrix. We call an element $M[r,c]$ an \emph{articulation} of $M$ if both $M[[r-1],[c-1]]$ and $M[[r+1,m],[c+1,n]]$ are empty.
\end{defn}
\begin{lemma}
\label{lemma3}
Let $M\in\Pat$, then for all $X\in\Mat$ it holds $X\in Cl(M)\Leftrightarrow$ there exists a sequence of articulations of $X$ such that each matrix in between two consecutive articulations of $X$ is a minor of $\smm{1}\nearrow M\nearrow\smm{1}$.  
\end{lemma}
\begin{proof}
\begin{itemize}
\item[$\Rightarrow$]
\item[$\Leftarrow$]
\end{itemize}
\end{proof}
\begin{thm}
For all $M\in\Pat$ there exists $\mathcal{F}$ finite such that $Cl(M)=Av(\mathcal{F})$.
\end{thm}
\begin{proof}
Using Lemma~\ref{lemma3}
\end{proof}
\begin{thm}
$Cl(\smm{1&0\\0&1})=Av\left(\smm{1&0&0\\0&0&1},\smm{1&0\\0&0\\0&1}\right)$.
\end{thm}
\begin{proof}
\begin{itemize}
\item[$\subseteq$]
\item[$\supseteq$]
\end{itemize}
\end{proof}
\begin{thm}
$Cl\left(\smm{1&0\\0&0\\0&1}\right)=Av\left(\smm{1&0&0\\0&0&1},\smm{1&0\\0&0\\0&0\\0&1},\smm{1&0\\1&0\\0&1},\smm{1&0\\0&1\\0&1}\right)$.
\end{thm}
We can generalize direct sum to allow the matrices to overlap.
\begin{defn}
TODO $A\oplus_{k\times l}B$
\end{defn}
\begin{thm}
Let $\mathcal{C}$ be any class of matrices such that
\begin{itemize}
\item $\mathcal{C}$ is closed under deleting of one-entries and
\item $\mathcal{C}$ is closed under the direct sum with $k\times l$ overlap and
\item there is any $M\in\Mat$ in $\mathcal{C}$
\end{itemize}
then $\mathcal{C}$ is also closed under direct sum with $m-2k\times n-2l$ overlap.
\end{thm}
\begin{proof}
Choose any two $A,B\in\mathcal{C}$ and $C\mathcal{C}$ such that $C\in\Mat$. Let $D\in\mathcal{C}$ denote the direct sum with $k\times l$ overlap of $A$ and $C$. Finally, let $E$ be the direct sum with $k\times l$ overlap of $D$ and $B$. It has the same size as $F$, the direct sum with $m-2k\times n-2l$ overlap of $A$ and $B$, which set of one-entries is also a subset of one-entries of $E\in\mathcal{C}$; therefore $F\in\mathcal{C}$.
\end{proof}
\begin{thm}
\label{hered}
Let $\mathcal{C}$ be any class of matrices that is hereditary according to interval minors then for all $m,n,k,l$ if $\mathcal{C}$ is closed under the direct sum with $m\times n$ overlap then is is also closed under the direct sum with $m+k\times n+l$ overlap.
\end{thm}
\begin{proof}
For contradiction, assume there are $A,B\in\mathcal{C}$ such that $A\oplus_{m+k\times n+l}B\not\in\mathcal{C}$.
%% do not really know why this should hold, need to think it through
\end{proof}
\begin{obs}
There is a $\mathcal{C}$ hereditary according to submatrices such that it is closed under the direct sum but it is not closed under the direct sum with $1\times1$ overlap.
\end{obs}
\begin{proof}
Let $\mathcal{C}$ be a class of all matrices obtained by applying the direct sum on $\smm{0&1\\1&0}$. Clearly, it is closed under the direct sum. On the other hand, $\smm{0&1\\1&0}\oplus_{1\times1}\smm{0&1\\1&0}=\smm{0&1&0\\1&0&1\\0&1&0}\not\in\mathcal{C}$.
\end{proof}
\begin{ntn}
We define Av($M$) to be a class of all matrices avoiding $M$ as 
%% as what? Should it be Av_\preceq?
\end{ntn}
We state following characterization only for the direct sum with $1\times1$ overlap but, because of Theorem~\ref{hered}, it also holds for any other size of overlap.
\begin{thm}
Let $M$ be a matrix. There are $M_1,M_2$ proper submatrices of $M$ such that $M=M_1\oplus_{1\times1}M_2\Leftrightarrow$ Av($M$) is not closed under the direct sum with $1\times1$ overlap.
\end{thm}
\begin{proof}
\begin{itemize}
\item[$\Rightarrow$]
\item[$\Leftarrow$]
\end{itemize}
\end{proof}
\begin{obs}
\label{obsart}
Let $M$ be a matrix. There are $M_1,M_2$ proper submatrices of $M$ such that $M=M_1\oplus_{1\times1}M_2\Leftrightarrow$ exists $r,c$ such that either
\begin{enumerate}
\item $M[r,c]$ is a one-entry and $(r,c)\in\{(1,1),(m,n)\}$ or
\item $M[r,c]$ is both top-right and bottom-left empty and $(r,c)\not\in\{(1,1),(m,n)\}$
\end{enumerate}
\end{obs}
\begin{defn}
Let $F$ be a matrix. We denote $\mathcal{R}(F)$ to be a set of all minimal (relating to minors) matrices $F'$ such that $F\im F'$ and $F'$ is not a direct sum with $1\times1$ overlap of proper submatrices of $F'$. For a class of matrices $\mathcal{F}$ let $\mathcal{R}(\mathcal{F})$ denote a set of all minimal (relating to minors) matrices from the set $\bigcup_{F\in\mathcal{F}}\mathcal{R}(F)$.
\end{defn}
\begin{thm}
\label{thmbasis}
Let $\mathcal{T}$ and $\mathcal{F}$ be classes of matrices such that $\mathcal{T}=Av(\mathcal{F})$, then $Cl(\mathcal{T})=Av(\mathcal{R}(\mathcal{F}))$.
\end{thm}
\begin{proof}Need to change the proof a bit probably after changing the statement
\begin{itemize}
\item[$\subseteq$] Instead of proving $M\in Cl(\mathcal{T})\Rightarrow M\in Av(\bigcup_{F\in\mathcal{F}}\mathcal{R}(F))$ we show $M\not\in Av(\bigcup_{F\in\mathcal{F}}\mathcal{R}(F))\Rightarrow M\not\in Cl(\mathcal{T})$. Assume $M\not\in Av(\bigcup_{F\in\mathcal{F}}\mathcal{R}(F))$. It follow from the definition that $M\in\bigcup_{F\in\mathcal{F}}\mathcal{R}(F)$, in particular, $M\in\mathcal{R}(F)$ for some $f\in\mathcal{F}$. Because of the definition of $\mathcal{R}(F)$, $M$ is not a direct sum with $1\times1$ overlap of proper submatrices of $M$ which means, according to Observation~\ref{obsart}, there are no non-trivial articulations and both top-right and bottom-left corners are empty. For contradiction, assume $M\in Cl(\mathcal{T})$, then, according to a generalization of Lemma~\ref{lemma3}, there exists a sequence of articulations of $M$ such that each matrix in between two consecutive articulations of $M$ is a minor of $\smm{1}\nearrow T\nearrow\smm{1}$ for some $T\in\mathcal{T}$. Since $M$ has only trivial articulations and they are both empty, it holds $M\im T$ and because of the choice of $M$ is also holds $M\im F$ for some $F\in\mathcal{F}$ which together give us a contradiction with $\mathcal{T}=Av(\mathcal{F})$.
\item[$\supseteq$] First of all, $Av(\bigcup_{F\in\mathcal{F}}\mathcal{R}(F))$ is closed under the direct sum with $1\times1$ overlap. For contradiction, assume there are $M_1,M_2\in Av(\bigcup_{F\in\mathcal{F}}\mathcal{R}(F))$ but $M=M_1\nearrow_{1}M_2\not\in Av(\bigcup_{F\in\mathcal{F}}\mathcal{R}(F))$. Then there exists $F'\in\mathcal{R}(F)$ for some $F\in\mathcal{F}$ such that $F'\im M$. Because $F'$ is not a direct sum with $1\times1$ overlap of proper submatrices of $F'$, it follows that either $F'\im M_1$ or $F'\im M_2$ and since $F\im F'$ we have a contradiction.

Now that we know that both sides are closed under the direct sum with $1\times1$ overlap it sufficient to show that the inclusion holds for any $M\in Av(\bigcup_{F\in\mathcal{F}}\mathcal{R}(F))$ that is not a direct sum with $1\times1$ overlap of proper submatrices of $M$. Such $M$ does not contain (again from Observation~\ref{obsart}) any non-trivial articulation and those trivial ones are empty. Because of that it holds $F\nim M$ for every $F\in\mathcal{F}$; otherwise either $M\in\mathcal{R}(F)$ or its minor would be there. Therefore $M\in\mathcal{T}$ and also $M\in Cl(\mathcal{T})$.
\end{itemize}
\end{proof}
\begin{defn}
Let $T$ be a class of matrices. The \emph{basis} of $T$ is a set of all minimal (relating to minors) matrices that do not belong to $T$.
\end{defn}
\begin{cor}
Let $\mathcal{T}$ and $\mathcal{F}$ be classes of matrices such that $\mathcal{T}=Av(\mathcal{F})$, then $\mathcal{R}(\mathcal{F})$ is a basis of $Cl(\mathcal{T})$.
\end{cor}
\begin{proof}
The proof follow directly from Theorem~\ref{thmbasis}.
\end{proof}
A natural question follows, whether the closure under direct sum of a class with finite basis has final basis. We prove that this is not the case.

\begin{defn}
Let $Nucleus_n$ be a matrix described by the examples below
\end{defn}
$Nucleus_1=\smm{ &1& },\ Nucleus_2=\smm{1&1& \\ &1&1},\ Nucleus_3=\smm{ &1&1& \\1& & &1\\ &1&1& },\ Nucleus_4=\smm{ & &1&1& \\ &1& & &1\\1& & &1& \\ &1&1& & },\ Nucleus_n=\smm{ & & & & & &1&1& \\ & & & & &1& & &1\\ & & & &1& & &1& \\ & & &\dots& & &1& & \\ & &1& & &\dots& & & \\ &1& & &1& & & & \\1& & &1& & & & & \\ &1&1& & & & & & \\}$
\begin{defn}
Let $Candy_{k,n,l}$ be a matrix given by $I_k\nearrow_{1\times2}Nucleus_n\nearrow_{1\times2}I_l$, where $I_k,I_l$ are unit matrices of sizes $k\times k$ and $l\times l$ respectively.
\end{defn}
%$\smm{
%  &   &   &   &   & 1 &   &   &   &   &   \\
%  &   &   &   &   &   & 1 &   &   &   &   \\
%  &   &   &   &   &   &   & 1 &   &   &   \\
%  &   &   &   &   &   &   &   & 1 &   &   \\
%  &   &   &   &   &   &   &   &   & 1 &   \\
%1 &   &   &   &   & 1 &   &   &   &   & 1 \\
%  & 1 &   &   &   &   &   &   &   &   &   \\
%  &   & 1 &   &   &   &   &   &   &   &   \\
%  &   &   & 1 &   &   &   &   &   &   &   \\
%  &   &   &   & 1 &   &   &   &   &   &   \\
%  &   &   &   &   & 1 &   &   &   &   & 
%}$
$Candy_{4,1,4}\smm{
  &   &   & 1 &   &   &   \\
  &   &   &   & 1 &   &   \\
  &   &   &   &   & 1 &   \\
1 &   &   & 1 &   &   & 1 \\
  & 1 &   &   &   &   &   \\
  &   & 1 &   &   &   &   \\
  &   &   & 1 &   &   & 
}$
%$\smm{
%  &   &   &   &   &   &   &   &   & 1 &   &   &   &   &   \\
%  &   &   &   &   &   &   &   &   &   & 1 &   &   &   &   \\
%  &   &   &   &   &   &   &   &   &   &   & 1 &   &   &   \\
%  &   &   &   &   &   &   &   &   &   &   &   & 1 &   &   \\
%  &   &   &   &   &   &   &   &   &   &   &   &   & 1 &   \\
%  &   &   &   &   &   &   &   & 1 & 1 &   &   &   &   & 1 \\
%  &   &   &   &   &   &   & 1 &   &   & 1 &   &   &   &   \\
%  &   &   &   &   &   & 1 &   &   & 1 &   &   &   &   &   \\
%  &   &   &   &   & 1 &   &   & 1 &   &   &   &   &   &   \\
%  &   &   &   & 1 &   &   & 1 &   &   &   &   &   &   &   \\
%1 &   &   &   &   & 1 & 1 &   &   &   &   &   &   &   &   \\
%  & 1 &   &   &   &   &   &   &   &   &   &   &   &   &   \\
%  &   & 1 &   &   &   &   &   &   &   &   &   &   &   &   \\
%  &   &   & 1 &   &   &   &   &   &   &   &   &   &   &   \\
%  &   &   &   & 1 &   &   &   &   &   &   &   &   &   &   \\
%  &   &   &   &   & 1 &   &   &   &   &   &   &   &   & 
%}$
$Candy_{4,4,4}\smm{
  &   &   &   &   & 1 &   &   &   \\
  &   &   &   &   &   & 1 &   &   \\
  &   &   &   &   &   &   & 1 &   \\
  &   &   &   & 1 & 1 &   &   & 1 \\
  &   &   & 1 &   &   & 1 &   &   \\
  &   & 1 &   &   & 1 &   &   &   \\
1 &   &   & 1 & 1 &   &   &   &   \\
  & 1 &   &   &   &   &   &   &   \\
  &   & 1 &   &   &   &   &   &   \\
  &   &   & 1 &   &   &   &   & 
}$
\begin{thm}
\label{thminf}
There exists a matrix $F$ such that $\mathcal{R}(F)$ is infinite.
\end{thm}
\begin{proof}
\end{proof}
\begin{cor}
There exists a class of matrices $\mathcal{C}$ having a finite basis such that $Cl(\mathcal{C})$ has an infinite basis.
\end{cor}
\begin{proof}
From Theorem~\ref{thminf}, we have a matrix $F$ for which $\mathcal{R}(F)$ is infinite. Let $\mathcal{C}=Av(F)$. Clearly, $\mathcal{C}$ has a finite basis. On the other hand, from Theorem~\ref{thmbasis} we have $Cl(\mathcal{C})=Av(\mathcal{R}(F))$ and $\mathcal{R}(F)$ is infinite from the choice of $F$.
\end{proof}
\chapter{Introduction}
Throughout the paper, every time we speak about matrices we mean binary matrices (also called 01-matrices) and we omit the word binary. If we speak about a \emph{pattern}, we again mean a binary matrix and we use the word in order to distinguish among more matrices as well as to indicate relationship.

When dealing with matrices, we always index rows and column starting with one and when we speak about a row~$r$, we simply mean a row with index $r$. A \emph{line} is a common word for both a row and a column. When we order a set of lines, we first put all rows and then all columns. For $M\in\Mat$, $[m]$ is a set of all rows and $[m+n]$ is a set of all lines, where $m$-th element is the last row. This goes with the usual notation.
\begin{ntn}
For $n\in\mathbb{N}$ let $[n]:=\{1,2,\dots,n\}$ and for $m\in\mathbb{N}$, where $n\leq m$ let $[n,m]:=\{n,n+1,\dots,m\}$.
\end{ntn}
\begin{ntn}
For a matrix $M\in\{0,1\}^{m\times n}$ and $L\subseteq[m+n]$ let $M[L]$ denote a submatrix of $M$ induced by lines in $L$.
\end{ntn}
\begin{ntn}
For a matrix $M\in\Mat$, $R\subseteq[m]$ and $C\subseteq[n]$ let $M[R,C]$ denote a submatrix of $M$ induced by rows in $R$ and columns in $C$. Furthermore, for $r\in[m]$ and $c\in[n]$ let $M[r,c]:=M[\{r\},\{c\}]=M[\{r,c+m\}]$.
\end{ntn}
\begin{defn}
We say a matrix $M\in\Mat$ \emph{contains} a pattern $P\in\Pat$ \emph{as a submatrix} and denote it by $\PsmM$ if there are $R\in[m]$ and $C\in[n]$ such that $|R|=k$, $|C|=l$ and for every $r\in R$ and $c\in C$ if $P[r,c]=1$, then $M[R,C][r,c]=1$.
\end{defn}
This does not necessarily mean $P=M[R,C]$ as $M[R,C]$ can have more one-entries than $P$ does.
\begin{ntn}
For a matrix $M\in\Mat$ and $L\subseteq[m+n]$ let $M_{\preceq}[L]$ denote a matrix acquired from $M$ by applying following operation for each $l\in L$:
\begin{itemize}
\item If $l$ is the first row in $L$ then we replace the first $l$ rows by one row that is a bitwise OR of replaced rows.
\item If $l$ is the first column in $L$ then we replace the first $l-m$ columns by one column that is a bitwise OR of replaced columns.
\item Otherwise, we take $l$'s predecessor $l'\in L$ in the standard ordering and replace lines $[l'+1,l]$ by one line that is a bitwise OR of replaced lines.
\end{itemize}
\end{ntn}
\begin{ntn}
For a matrix $M\in\Mat$, $R\subseteq[m]$ and $C\subseteq[n]$ let $M_{\preceq}[R,C]:=M_{\preceq}[R\cup \{c+m|c\in C\}]$.
\end{ntn}
\begin{defn}
We say a matrix $M\in\Mat$ \emph{contains} a pattern $P\in\Pat$ \emph{as an interval minor} and denote it by $\PimM$ if there are $R\in[m]$ and $C\in[n]$ such that $|R|=k$, $|C|=l$ and for every $r\in R$ and $c\in C$ if $P[r,c]=1$, then $M_{\preceq}[R,C][r,c]=1$.
\end{defn}
%ordered bipartite graphs
%
%
%
%partitioning into rectangles (interval minors)
\begin{obs}
\label{obs1}
For all matrices $M$ and $P$, $\PsmM\Rightarrow\PimM$.
\end{obs}
\begin{obs}
For all matrices $M$ and $P$, if $P$ is a permutation matrix, then $\PsmM\Leftrightarrow\PimM$.
\end{obs}
\begin{proof}
If we have $\PimM$, then there is a partitioning of $M$ into rectangles and for each one-entry of $P$ there is at least one one-entry in the corresponding rectangle of $M$. Since $P$ is a permutation matrix, it is sufficient to take rows and columns having at least one one-entry in the right rectangle and we can always do so.

Together with Observation~\ref{obs1} this gives us the statement.
\end{proof}
\begin{obs}
Let $M\in\Mat$ and $P\in\Pat$, $\PimM\Leftrightarrow P^T\im M^T$.
\end{obs}
Because of this observation we will usually only show results only for rows or columns and expect both to hold and only show results for $P\in\Pat$ but assume the symmetrical results for $P^T$.
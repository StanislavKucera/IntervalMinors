\newsavebox{\smlmat}
\savebox{\smlmat}{$\smm{1&1\\1&0}$}
\newsavebox{\smlmatb}
\savebox{\smlmatb}{$\smm{1&1\\1&1}$}
\newsavebox{\smlmatc}
\savebox{\smlmatc}{$\smm{1&1&1\\0&1&0}$}

\chapter{Characterizations}
\begin{defn}
A \emph{walk} in a matrix~$M$ is a sequence of some of its entries, beginning in the top left corner and ending in the bottom right one. If an entry $M[i,j]$ is in the sequence, the next one is either $M[i+1,j]$ or $M[i,j+1]$.
\end{defn}
\begin{defn}
We call a binary matrix~$M$ a \emph{walking matrix} if there is a walk in $M$ such that all one-entries of $M$ are contained on the walk.
\end{defn}
\begin{defn}
An \emph{extended walk of size $k\times l$} in a matrix~$M$ is a subset of some of its entries, beginning in the top left corner and ending in the bottom right one. If an entry $M[i,j]$ is in the subset there is also either $M[i+1,j]$ or $M[i,j+1]$. The size describes that no more than $k$ entries directly above each other are in the subset and no more than $l$ entries directly next to each other are in the subset. We say that an extended walk of size $k\times l$ in $M$ starts with a walk~$w$, if the extended walk is a subset of entries of $M$ that
\begin{itemize}
\item lie on $w$ or below $w$ and
\item lie on $w$ shifted by $k-1$ down and by $l-1$ to the left or above it.
\end{itemize}
\end{defn}
\begin{defn}
For $M\in\Mat$ and $r\in[m],c\in[n]$ we say $M[r,c]$ is
\begin{itemize}
\item \emph{top-left empty} if $M[[r-1],[c-1]]$ is an empty matrix,
\item \emph{top-right empty} if $M[[r-1],[c+1,n]]$ is empty,
\item \emph{bottom-left empty} if $M[[r-1],[c+1,n]]$ is empty,
\item \emph{bottom-right empty} if $M[[r-1],[c+1,n]]$ is empty.
\end{itemize}
\end{defn}
\section{Patterns of size $2\times2$ and their generalization}
\begin{thm}
\label{walkingthm}
Let $P=\smm{0&1\\1&0}$, then for all $M$: $\PnimM\Leftrightarrow M$ is a walking matrix.
\end{thm}
\begin{proof}
Since $P$ is a permutation matrix, $\PnimM\Leftrightarrow\PnsmM$ and it is easy to see $\PnsmM\Leftrightarrow M$ is a walking matrix.
\end{proof}

Now consider a generalization of the pattern from above:
\begin{thm}
Let $P\in\Pat$ be a matrix having only two one-entries -- $P[1,n]$ and $P[m,1]$, then for all $M$: $\PnimM\Leftrightarrow M$ has an extended walk of size $k-1\times l-1$ containing all one-entries.
\end{thm}
\begin{proof}
\begin{itemize}
\item[$\Rightarrow$] Let $\PnimM$ and consider the left-most top-right empty elements of $M$. They necessarily form a walk $w$. For contradiction, assume there is a one-entry $e$ below the extended walk of size $k-1\times l-1$ starting with $w$. Since $e$ is below the extended walk, there is an element $e'$ - the right-most element of $M$ that is neither below $e$ nor to the right from $e$ and at the same time still below the extended walk (it is possible $e=e'$). Let $e=M[r,c]$ and notice $M[r-k,c-l]$ is part of walk $w$ and because of the choice of $e'$ neither $M[r-k-1,c-l]$ nor $M[r-k,c-l-1]$ are on the walk $w$ and $M[r-k,c-l]$ must be a one-entry; therefore, together with $e$ it forms the forbidden pattern in $M$, which is a contradiction.
\item[$\Leftarrow$] Let $M[r,c]$ be any one-entry of $M$, which then necessarily lie in the extended walk. Because the size of the walk is $k-1\times l-1$, $M[r-k+1,c-l+1]$ is top-left empty and $M[r+k-1,c+l-1]$ is bottom-right empty; therefore $e$ cannot be a part of a mapping of $P$. 
\end{itemize}
\end{proof}

\begin{thm}
\label{theorem1}
Let $P=\smm{1&1\\1&0}$, then for all $M\in\Mat$: $\PnimM\Leftrightarrow$ there exist a row~$r$ and a column~$c$ such that (see Figure~\ref{p12})
\begin{itemize}
\item $M[[r-1],[c-1]]$ is empty,
\item $M[[r-1],[c+1,n]]$ is empty,
\item $M[[r+1,m],[c-1]]$ is empty and
\item $M[[r,m],[c,n]]$ is a walking matrix.
\end{itemize}
\end{thm}
\begin{figure}[h!]
\centering
\includegraphics[width=60mm]{img/p12.pdf}
\caption{Characterization of a matrix avoiding \usebox{\smlmat} as an interval minor. Matrix $M'$ is a walking matrix}
\label{p12}
\end{figure}
\begin{proof}
\begin{itemize}
\item[$\Rightarrow$] If $\smm{0&1\\1&0}\nim M$ then $M$ is a walking matrix and we set $r=c=1$. Otherwise, there are one-entries $M[r,c']$ and $M[r',c]$ such that $r'<r$ and $c'<c$. If there is a one-entry in regions $M[[r-1],[c-1]],\ M[[r-1],\ [c+1,n]]$ or $M[[r+1,m],[c-1]]$ then $\PimM$. If $M[[r,m],[c,n]]$ is not a walking matrix then it contains $\smm{0&1\\1&0}$ and we again get a contradiction.
\item[$\Leftarrow$] For contradiction, assume that $M$ described in Figure~\ref{p12} contains $P$ as an interval minor. It means that there is a partition of the matrix into four quadrants such that there is at least one one-entry in each quadrant besides the bottom right one. If the matrix is partitioned above the $r$-th row, then there is only one column containing one-entries and it is not possible for both top quadrants to have a one-entry. Similarly, if the matrix is partitioned to the left of the $c$-th column, there is only one row containing one-entries and there is no one-entry in either top-left or bottom-left quadrant. Therefore, the partitioning lies bellow the $r$-th row and to the right of the $c$-th column, but if the quadrants contain one-entries, there is a $\smm{0&1\\1&0}$ interval minor in $M'$, which is a contradiction with it being a walking matrix. % there has to be a way to write this better
\end{itemize}
\end{proof}

\begin{thm}
Let $P\in\Pat$ be a matrix having only three one-entries -- $P[1,1],\ P[1,n]$ and $P[m,1]$, then for all $M$: $\PnimM\Leftrightarrow$ there exist a row~$r$ and a column~$c$ such that (see Figure~\ref{p12} and imagine rows and columns being extended)
\begin{itemize}
\item $M[[r-1],[c-1]]$ is empty,
\item $M[[r-1],[c+l,n]]$ is empty,
\item $M[[r+k,m],[c-1]]$ is empty and
\item $M[[r,m],[c,n]]$ has an extended walk of size $k-1\times l-1$ containing all one-entries.
\end{itemize}
\end{thm}
\begin{proof} Let $P'=P$ and set $P'[m,1]=0$ ($P'$ is a generalization of $\smm{0&1\\1&0}$). 
\begin{itemize}
\item[$\Rightarrow$] If $P'\nim M$ then $M$ is a matrix having an extended walk of size $k-1\times l-1$ containing all one-entries and we set $r=c=1$. Otherwise, there are one-entries $M[r_1,c_1]$ and $M[r_2,c_2]$ such that $r_2<r_1$ and $c_1<c_2$. We now choose $M[r_3,c_3]$ to be the bottom-most one-entry that still forms $P'$ with $M[r_2,c_2]$. We choose $M[r_4,c_4]$ to be the left-most one-entry that forms $P'$ with $M[r_3,c_3]$ and set $r=r_3-k+1$ and $c=c_4-l+1$. If there is a one-entry in regions $M[[r-1],[c-1]],\ M[[r-1],\ [c+l,n]]$ or $M[[r+k,m],[c-1]]$ then $\PimM$. If $M[[r,m],[c,n]]$ is not a walking matrix then it contains $P'$ and we again get a contradiction.
\item[$\Leftarrow$] Because of the sizes of areas with no one-entries and the condition for $M[[r,m],[c,n]]$, there cannot be $P'$ anywhere but in $M[[r+k-1],[c+l-1]]$. Since $M[[r-1][c-1]]$ is empty, there is no one-entry to map $P[1,1]$ to; therefore, $\PnimM$.
\end{itemize}
\end{proof}

\begin{lemma}
\label{lemma1}
Let $P=\smm{1&1\\1&1}$ and let $M\in\Mat$ avoid $P$ as an interval minor, then there exists a row~$r$ and a column~$c$ such that $M[r,c]$ is either
\begin{enumerate}
\item a one-entry and $(r,c)\in\{(1,1),(1,n),(m,1),(m,n)\}$ or
\item both top-left empty and bottom-right empty and $(r,c)\not\in\{(1,n),(m,1)\}$ or
\item both top-right empty and bottom-left empty and $(r,c)\not\in\{(1,1),(m,n)\}$.
\end{enumerate}
\end{lemma}
\begin{proof}
If there is a one-entry in any corner we are done. Otherwise, let $A$ be a set of all top-left empty entries of $M$ and $B$ be a set of all bottom-right empty entries of $M$. If there is an entry $M[r,c]\in A\cap B$ different from $(1,n)$ and $(m,1)$ we are done. Assume $A\cap B=\{(1,n),(m,1)\}$. Since $(m,1)\in A$, it also holds $(m-1,1)\in A$ and because it is not in the intersection we have $(m-1,1)\not\in B$. This means $M[m-1,1]$ is not bottom-right empty; therefore there is a one-entry somewhere in $M[{m},[2,n]]$. Moreover, no corner contains a one-entry so the is a one-entry in $M[{m},[2,n-1]]$. For simplicity, we will say that the last row in non-empty (knowing the corners are empty). Symmetrically, we also get that the first row is non-empty and both the first and the last columns are non-empty. If there is a one-entry $M[r_l,1]$ in a different row than a one-entry $M[r_r,n]$ and at the same time a one-entry $M[1,c_t]$ in a different column than a one-entry $M[m,c_b]$ then these four one-entries form a mapping of the forbidden pattern $P$.

This is not true!!!
%% the last statement is not true -- need to fix it, there is one more case and symmetry - will use a picture to deal with it

Without loss of generality assume there is only one one-entry in both the first and the last column and they are both in the same row $r'$. Let $c'$ be a column such that there is a one-entry $M[1,c']$. Clearly, there is no other column that contains a one-entry above $r'$, because we would again get a contradiction. Symmetrically, let $c''$ be the only column containing one-entries below $r'$. If $c'\geq c''$ we have that both $M[r',c']$ and $M[r',c'']$ are both top-left empty and bottom-right empty, which is a contradiction with $A\cap B=\{(1,n),(m,1)\}$. Otherwise, $c'<c''$ and both $M[r',c']$ and $M[r',c'']$ are both top-right empty and bottom-left empty where $(r',c')\not\in\{(1,1),(m,n)\}$ which concludes the proof.
\end{proof}

\begin{thm}
\label{t33}
Let $P=\smm{1&1\\1&1}$, then for all $M$: $\PnimM\Leftrightarrow M$ looks like one of the matrices in Figure~\ref{p33}, where $\smm{1&1\\1&0}\nim M_1$, $\smm{0&1\\1&1}\nim M_2$, $\smm{1&1\\0&1}\nim M_3$ and $\smm{1&0\\1&1}\nim M_4$.
\end{thm}
\begin{figure}[h!]
\centering
\includegraphics[height=60mm]{img/p33.pdf}
\caption{Characterization of a matrix avoiding \usebox{\smlmatb} as an interval minor.}
\label{p33}
\end{figure}
\begin{proof}
\item[$\Rightarrow$] We proceed by induction by the size of $M$.

If $M\in\{0,1\}^{2\times2}$ then it either avoids $\smm{0&1\\1&1}$ or $\smm{1&1\\1&0}$ and we are done.

For bigger $M$ there is, from Lemma~\ref{lemma1}, $M[r,c]$ satisfying some conditions. If it is the first condition -- there is a one-entry in any corner, we are done because the matrix cannot contain one of the rotations of $\smm{1&1\\1&0}$. Assume the second case -- $M[r,c]$ is both top-right and bottom-left empty and $(r,c)\not\in\{(1,n),(m,1)\}$. If $M_1$ is non-empty, then $\smm{0&1\\1&1}\nim M_2$; otherwise, $\PimM$. Similarly, $\smm{1&1\\1&0}\nim M_1$ if $M_2$ is non-empty. If one of them is empty, the other is a smaller matrix avoiding $P$ as an interval minor and by induction hypothesis, it can be partitioned. Adding empty rows and columns does not break any condition and we get a partitioning of the whole $M$.
\item[$\Leftarrow$] Without loss of generality, let us assume $M$ looks like the left matrix in Figure~\ref{p33}. For contradiction, assume $\PimM$. In that case, we can partition $M$ into four quadrants such that there is at least one one-entry in each of them. It does not matter where we partition it, every time we either get $\smm{1&1\\1&0}\im M_1$ or $\smm{0&1\\1&1}\im M_2$, which is a contradiction.
\end{proof}

\begin{thm}
Let $P\in\Pat$ be a matrix having only four one-entries -- $P[1,1],\ P[1,n],\ P[m,1]$ and $P[m,n]$, then for all $M$: $\PnimM\Leftrightarrow M$ looks like one of the matrices in Figure~\ref{p33}, where generalized $\smm{1&1\\1&0}\nim M_1$, $\smm{0&1\\1&1}\nim M_2$, $\smm{1&1\\0&1}\nim M_3$ and $\smm{1&0\\1&1}\nim M_4$.
\end{thm}
% a lot to do here - probably will need a generalization of the lemma1

\section{Matrices of size $2\times3$}
\begin{thm}
Let $P=\smm{1&0&1\\0&1&0}$, then for all $M$: $\PnimM\Leftrightarrow M=M_1\oplus_hM_2$ where $\smm{1&0\\0&1}\nim M_1$ and $\smm{0&1\\1&0}\nim M_2$.
\end{thm}
\begin{proof}
\begin{itemize}
\item[$\Rightarrow$] Let $e=[r,c]$ be the top-most one-entry of $M$. If $\smm{1&0\\0&1}\im M[[m],[c-1]]$, together with $e$ it forms $P$. If $\smm{0&1\\1&0}\nim M[[m],[c,n]]$ then we are done. Let us assume it is not the case and let $e_{0,0},\ e_{1,1}$ be any two one-entries forming the forbidden pattern. Symmetrically, let $\smm{1&0\\0&1}\im M[[m],[c]]$ and let $e_{0,1},\ e_{1,0}$ be any two one-entries forming the forbidden pattern. Now if we take $e_{0,0},\ e_{0,1}$ and $e_{1,0}$ or $e_{1,1}$ with bigger row, we get the forbidden pattern $P$ as an interval minor of $M$. 
\item[$\Leftarrow$] For contradiction, let us assume $\PimM$ and $M=M_1\oplus_hM_2$. If $\PimM$, look at the one-entry of $M$ where the bottom one-entry of $P$ is mapped. If it is in $M_1$ then $\PnimM$ because $\smm{0&1\\1&0}\nim M_1$. Otherwise, $\PnimM$ because $\smm{1&0\\0&1}\nim M_2$.
\end{itemize}
\end{proof}
\begin{lemma}
\label{lemma2}
Let $P=\smm{1&1&1\\0&1&0}$, then for all $M$: $\PnimM\Rightarrow M=M_1\oplus_hM_2$ where
\begin{enumerate}
\item $\smm{1&1\\0&1}\nim M_1$ and $\smm{0&1\\1&0}\nim M_2$ or
\item $\smm{1&0\\0&1}\nim M_1$ and $\smm{1&1\\1&0}\nim M_2$.
\end{enumerate}
\end{lemma}
\begin{proof}
Let $e=[r,c]$ be the top-most one-entry of $M$. If $\smm{1&1\\0&1}\im M[[m],[c-1]]$, together with $e$ it would be the whole $P$. Similarly, $\smm{1&1\\1&0}\nim M[[m],[c+1,n]]$. For contradiction with the statement, let $\smm{1&0\\0&1}\im M[[m],[c]]$ and $e_{0,0},\ e_{1,1}$ (none of them equal to $e$, since $e$ lies in the top-right corner) be any two one-entries forming the pattern. Symmetrically, let $\smm{0&1\\1&0}\im M[[m],[c,n]]$ and $e_{0,1},\ e_{1,0}$ be any two one-entries forming the pattern. In that case $e_{0,0},\ e,\ e_{0,1}$ and $e_{1,0}$ or $e_{1,1}$ with bigger row give us the forbidden pattern $P$ as an interval minor of $M$.
\end{proof}
\begin{thm}
Let $P=\smm{1&1&1\\0&1&0}$, then for all $M$: $\PnimM\Leftrightarrow M$ looks like the matrix in Figure~\ref{p72} and $\smm{1&0\\0&1}\nim M_1$ and $\smm{0&1\\1&0}\nim M_2$.
\end{thm}
\begin{figure}[h!]
\centering
\includegraphics[height=60mm]{img/p72.pdf}
\caption{Characterization of a matrix avoiding \usebox{\smlmatb} as an interval minor.}
\label{p72}
\end{figure}
\begin{proof}
\begin{itemize}
\item[$\Rightarrow$] From Lemma~\ref{lemma2} we know $M=M_1'\oplus_hM_2'$ where $\smm{1&1\\0&1}\nim M_1'$ and $\smm{0&1\\1&0}\nim M_2'$. The second case would be dealt with symmetrically. From Theorem~\ref{theorem1} we have that $M_1'$ can be characterized exactly like $M[[m],[c_2-1]$ and $M[[m],[c_2,n]]$ forms a walking matrix. The only problem with our claim would be if there were two different columns having a one-entry above the $r$-th row. In that case, those two one-entries together with a one-entry in the $r$-th row between the columns $c_1$ and $c_2$ and a one-entry in the $c_1$-th column above the $r$-th row form $P$ as an interval minor.
\item[$\Leftarrow$] The bottom-middle one-entry of $P$ can not be mapped anywhere but to the $r$-th row, but in that case there are at most two columns having one-entries above it. %(will do better hopefully)
\end{itemize}
\end{proof}
\begin{thm}
Let $P=\smm{0&1&1\\1&0&0}$, then for all $M$: $\PnimM\Leftrightarrow M$ contains a walk $w$, no one-entries below the walk and for each entry $M[r,c]$ of the walk there is at most one non-empty column in $M[[r-1],[c+1,n]]$.
\end{thm}
\begin{proof}
\begin{itemize}
\item[$\Rightarrow$] Let $w$ be any walk containing all the top-most and right-most entries that are bottom-left empty. From the choice of $w$, there are no one-entries below it and if all $M[r,c]$, $M[r-1,c]$ and $M[r,c+1]$ are on $w$ then $M[r,c]$ is a one-entry as else $M[r,c]$ was neither top-most nor right-most bottom-left empty. As a consequence, whenever we choose $M[r,c]$ from $w$, it either is a one-entry or there is one-entry in the same row to the left of it. For contradiction let us now assume that there is an entry of the walk $M[r,c]$ for which there are two non-empty columns in $M[[r-1],[c+1,m]]$. Then a one-entry from each of those columns and a one-entry in $M[r,c]$ or to the left of it together give us $\PimM$ and consequently a contradiction. 
\item[$\Leftarrow$] For contradiction let $\PimM$. Without loss of generality we can assume that the bottom-left entry of $P$ is mapped somewhere to the walk -- to $M[r,c]$. But then $\smm{1&1}\im M[[r-1],[c+1,n]]$ which is a contradiction with it having one-entries in at most one column.
\end{itemize}
\end{proof}

\section{Empty rows and columns}
\begin{obs}
\label{emptyrows}
Let $P\in\Pat$ and $P'\in\{0,1\}^{k\times l+1}$ such that $P'=P\oplus_h0^{k\times1}$, similarly let $M\in\Mat$ and $M'\in\{0,1\}^{m\times n+1}$ such that $M'=M\oplus_h1^{m\times1}$, then $\PimM\Leftrightarrow P'\im M'$.
\end{obs}
\begin{proof}
\begin{itemize}
\item[$\Rightarrow$] Clearly we can map the last column of $P'$ to the last column of $M'$ and then map (using OR) $P'[[k],[l]]$ to $M'[[m],[n]]$ the same way $P$ is mapped to $M$.
\item[$\Leftarrow$] If $P'\im M$ we are done. Otherwise, the last column of $P'$ needs to be mapped to the last column of $M'$ and by deleting both from their matrix we get $P'[[k],[l]]\im M'[[m],[n]]$ which is the same as $\PimM$.
\end{itemize}
\end{proof}

The same proof can be also used for adding an empty column as the first column or an empty row as the first or the last row. Using induction we can easily show that a pattern $P'$ is avoided by a matrix $M'$ if and only if $P$ is avoided by $M$ where $P$ is derived from $P'$ by excluding all empty beginning or ending rows and columns and $M$ is derived from $M'$ by excluding the same number of beginning or ending rows and columns. Therefore, when characterizing matrices avoiding a forbidden pattern, we do not need to consider patterns having empty rows or columns on their boundary.

For the following two statements, let $P\in\{0,1\}^{k\times2}$ be a forbidden pattern and $P^+\in\{0,1\}^{k\times3}$ be the pattern created from $P$ by adding a new column in between the two columns of $P$.
\begin{lemma}
\label{lemmamax}
Let $M\in Av(P^+)$ be an inclusion maximal matrix, then each row of $M$ either contains no one-entries or exactly one interval of one-entries of length at least 2.
\end{lemma}
\begin{proof}
If there are two one-entries on the same row and there is a zero-entry in between them, we can change the zero-entry into a one-entry and the new matrix will still avoid $P^+$.

If there is only one one-entry in a row, we can always change one of its neighbors into a one-entry. If the one-entry lies in the first column, we can insert another one-entry to the second column and the matrix will still avoid $P^+$, because if the new one-entry is used in a mapping of $P^+$ then it must as a part of the first column and we could use the one-entry to its left instead. Similarly, there is no one-entry in the last column that is the only one in it's row. For contradiction, assume there is the only one one-entry in a row, call it $e=M[r,c]$, that is not in the first nor in the last column and assume we cannot change neither $e_l=M[r,c-1]$ nor $e_r=M[r,c+1]$ to a one-entry, because then the matrix would contain the minor $P^+$. Clearly, all mappings of $P^+$ that are created by a change of $e_1$ contain $e_1$ and the element of $P^+$ that is mapped to $e_1$ comes from the first column and is on $r_1$-th row (there might be multiple of them so just take any), because if it came from the middle row, we could have used the original zero-entry instead as the middle columns is empty and if it came from the last row, we could have used $e$ instead. Similarly, $P^+[r_2,3]$ is mapped to $e_2$ when $e_2$ is changed into a one-entry. We now want to show that the two partitionings can be altered to have a mapping of $P^+$ to $M$. We can always map the middle column of $P^+$ to the single column where $e$ is presented and we describe how to partition $M$ into $k$ rows. We have two cases to go through:
\begin{itemize}
\item[$r_1\neq r_2$] TODO
\item[$r_1=r_2$] TODO
\end{itemize}
\end{proof}
TODO generalize the result for addition of multiple empty columns
\begin{thm}
\label{emptymiddle}
For all $M\in\Mat$ it holds $M\in Av(P^+)\Leftrightarrow$ there exists $N\in\{0,1\}^{m\times(n-1)}$ such that $N\in Av(P)$ is inclusion maximal and $M$ is a submatrix of $N\oplus_h0^{m\times1}$ placed over $0^{m\times1}\oplus_hN$ with an operation bitwise OR.
\end{thm}
\begin{proof}
\begin{itemize}
\item[$\Rightarrow$] It suffices to only prove the statement for $M$ that is inclusion maximal. To do so, we use Lemma~\ref{lemmamax}. It say that each row of $M$ contains either no one-entry or an interval of length at least two. From that we define $N$ to be created from $M$ by deleting the last one-entry on each row and excluding the last column. Clearly, $M$ is equal to $N\oplus_h0^{m\times1}$ placed over $0^{m\times1}\oplus_hN$ with an operation bitwise OR. If $P\im N$ then each mapping of $P$ can be extended to a mapping of $P^+$ to $M$ by ... How to say this?

TODO
\item[$\Leftarrow$] It suffices to show that $M$ that is equal to $N\oplus_h0^{m\times1}$ placed over $0^{m\times1}\oplus_hN$ with an operation bitwise OR belongs to $Av(P^+)$. For contradiction, assume it does not. Then there is mapping of $P^+$ into elements of $M$ and we can assume that one-entries of the first column of $P^+$ are mapped to those one-entries of $M$ created from $N\oplus_h0^{m\times1}$. If it was not the case and there was a one-entry mapped to a one-entry of $M$ created only from $0^{m\times1}\oplus_hN$ we can take an element directly to its left and that is created from $N\oplus_h0^{m\times1}$. Symmetrically, all one-entries of the last column of $P^+$ are mapped to one-entries created from $0^{m\times1}\oplus_hN$. ...

TODO
\end{itemize}
\end{proof}

\begin{lemma}
Let $P\in\Pat$ and let $M\in\Mat$ be an inclusion maximal matrix such that $\PimM$, then each row of $M$ contains at most $l-1$ intervals of one-entries and each column of $M$ contains at most $k-1$ intervals of one-entries.
\end{lemma}
\begin{proof}
It suffices to prove the statement only for rows. Let us proceed by contradiction and let us have a row~$r$ that contains at least $l$ intervals of one-entries. Let $Z_1$ denote the interval of zero-entries following the first interval of one-entries and similarly $Z_i$ denote the interval of zero-entries following the $i$-th interval of one-entries for $i\in[k]$. Because $M$ is inclusion maximal, each change of a zero-entry to a one-entry will create a copy of the forbidden minor and the new one-entry will be a part of each such copy.

We will show that changing a zero-entry from $Z_i$ can only create a copy where the changed entry is part of the $i+1$ or higher column. We proceed by induction:
\begin{itemize}
\item[$i=1$] For contradiction assume that changing any $e\in Z_1$ creates a copy of minor $P$ where some element of the first column of $P$ is mapped to $e$. In this case, because $Z_1$ follows after an interval of one-entries, any one-entry lying before $e$ on the same row can play the same role in the mapping and we have $\PimM$ which is a contradiction. 
\item[$i>1$] From the induction hypothesis, we know that changing a zero-entry of $Z_{i-1}$ will create a copy of $P$ where the changed element is used to map the $i$-th or further column of $P$. In particular, the first $i-1$ columns can always be mapped (even without changing any entry) to columns preceding $Z_{i-1}$ and therefore preceding $Z_{i}$. This means that if a change of a zero-entry of $Z_i$ introduces a mapping of $P$ that uses the new one-entry to map any of columns $1$ to $i-1$, we can combine the mapping with the fact that the first $i-1$ columns can be mapped before $Z_i$ and find that $\PimM$ in the following way:

Similarly, we are done if any change of an element~$e'$ of $Z_{i-1}$ is used to map the $i+1$ or higher column of $P$; therefore, we assume each such change only allows $P$ to map the $i$-th column there. The last case we need to take care of is when the change of an element $e\in Z_i$ creates a mapping of $P$ where the $i$-th column uses $e$. Let $r$ denote a row of $P$ that is mapped to $e$ when $e$ is a one-entry and let $r'$ denote a row of $P$ that is mapped to $e'$ when $e'$ is a one-entry.
\begin{itemize}
\item[$r=r'$] ... This is really hard to describe, probably will use a picture ... In this case, we take the partitioning created by both mappings and extend it to a partitioning of mapping $P$ into $M$ without $e$ or $e'$ being one-entries. We simply
\end{itemize}
\end{itemize}
\end{proof}

\section{Pattern size constrains}
In the previous sections, it always holds that for a pattern~$P$ of size $k\times l$ any inclusion maximal matrix $M$ that avoids $P$ as an interval minor has at most $l-1$ intervals of one-entries in each row and at most $k-1$ intervals of one-entries in each column. Is this a global phenomenon?

\begin{thm}
Let $P\in\{0,1\}^{k\times2}$ and $M\in\Mat$ be an inclusion maximal matrix such that $\PimM$, then $M$ contains at most one interval of one-entries in each row.
\end{thm}
\begin{proof}
For contradiction, assume there are several intervals of one-entries in the same row of $M$. Because $M$ is inclusion maximal still avoiding $P$ as an interval minor, changing any one-entry~$e$ in between two consequtive intervals of one-entries will create a mapping of the forbidden pattern. Such a mapping uses the changed one-entry to map some element $P[r',1]$ or $P[r',0]$. In the first case, the same mapping will work if we use any one-entry that is to the left from $e$ instead of $e$, which would lead to $\PimM$ and therefore cannot be. In the second case, the mapping can use any one-entry to the right from $e$ instead of $e$ and therefore, there is no mapping of $P$ to $M$ even if $e$ is a one-entry. That is a contradiction with $M$ being inclusion maximal.
\end{proof}
\begin{thm}
\label{smallintnum}
Let $P\in\{0,1\}^{2\times k}$ and $M\in\Mat$ be an inclusion maximal matrix such that $\PimM$, then $M$ contains at most $O(k^2)$ intervals of one-entries in each row.
\end{thm}
\begin{proof}
Let us fix a row of a big enough inclusion maximal matrix $M$ and let it have as many intervals of one-entries as possible without creating a mapping of $P$.

TODO
\end{proof}
We see that for patterns having only two rows or columns we can indeed find a structural property of any inclusion maximal matrix avoiding it. On the other hand, already for a pattern of size $3\times3$ we show that there are matrices with arbitrarily many intervals of one-entries and we cannot get their number smaller.
\begin{thm}
\label{manyints}
Let $P=\smm{ &1& \\1& & \\ & &1}$. For every $k>1$ there is an inclusion maximal matrix $M$ avoiding $P$ as an interval minor having $k$ intervals of one-entries.
!Need to check this really is an inclusion maximal matrix or restate!
\end{thm}
\begin{proof} Let $M$ be the $(2k-1)\times(2k-1)$ matrix described by the picture:
$$\smm{	1&0&1&0&1&\cdots&1&0&1&0&1\\
		 & & & & &\cdots& & &1&1&1\\
		 & & & & &\cdots& & &1&1&1\\
		 & & & & &\cdots&1&1&1& & \\
		 & & & & &\cdots&1&1&1& & \\
		\vdots&\vdots&\vdots&\vdots&\vdots&\iddots&\vdots&\vdots&\vdots&\vdots&\vdots\\
		 & &1&1&1&\cdots& & & & & \\
		 & &1&1&1&\cdots& & & & & \\
		1&1&1& & &\cdots& & & & & \\
		1&1&1& & &\cdots& & & & & \\
		 }$$
$\PnimM$ because we always need to map $P[2,1]$ and $P[3,3]$ to just one ``block'' of one-entries of $M$ which only leaves zero-entries where we need to map $P[1,2]$.

When we change any zero-entry of the first row into a one-entry (getting $M'$), not only $P\im M'$ but also any $P'\in\{0,1\}^{3\times3}$ such that $P\im P'$ will be the minor of $M'$.
\end{proof}
We have seen in the proof that already for matrices of size $3\times3$ there is a big portion of those patterns for which we can create an inclusion maximal matrix with many intervals of one-entries. Also, the pattern $P$ has a specific structure and we can extend the result.
\begin{thm}
Let $P=\smm{ &1& \\1& & \\ & &1}$. For every $P'$ such that $P\im P$ and every $k>1$ there is an inclusion maximal matrix $M$ avoiding $P'$ as an interval minor having $k$ intervals of one-entries.
\end{thm}
\begin{proof}
First, assume there is a mapping of $P$ into $P'\in\Pat$ that assigns a one-entry of the first row to $P[1,2]$, a one-entry of the first column to $P[2,1]$ and a one-entry of the last row and column to $P[3,3]$. Then, we can construct a similar matrix as we did in the proof of Theorem~\ref{manyints} avoiding $P'$ but after changing one zero-entry of the first row it contains the whole $\{1\}^{k\times l}$.

Let $P'$ be any pattern containing $P$ without additional restrictions. Let $P'[r_1,c_1],P'[r_2,c_2]$ and $P'[r_3,c_3]$ be one-entries of $P'$ that can be used to map $P[1,2],P[2,1]$ and $P[3,3]$ to it respectively. Then we take a submatrix $P'':=P'[[r_1,r_3],[c_2,c_3]]$. Such a matrix fulfills assumptions of the more restricted case stated at the beginning of the proof and we can find an inclusion maximal matrix $M'$ avoiding $P''$ having $k$ intervals. We construct $M$ from $M'$ by simply adding new rows and columns, all containing one-entries. We add $r_1-1$ rows in front of the first row and $k-r_3$ rows behind the last row. We also add $c_2-1$ columns in front of the first column and $l-c_3$ columns behind the last column. Constructed matrix $M$ avoids pattern $P'$ as its submatrix $P''$ cannot be mapped to $M$ without added rows and columns. At the same time, any change of a zero-entry of the $r_1$-th row of $M$ to one-entry creates a copy of ${1}^{k\times l}$ so the changed matrix contains $P'$.
\end{proof}
What makes it more interesting is that for any pattern that avoids any rotation of $P$ from the previous theorem, every inclusion maximal matrix avoiding it has a bounded number of intervals of one-entries in both rows and columns. To prove that we need a few partial results.
\begin{thm}
Let $P$ be a pattern avoiding all rotations of $P_1=\smm{ &1& \\1& & \\ & &1}$, then $P$:
\begin{enumerate}
\item contains one non-empty line or
\item contains two non-empty lines or
\item contains three non-empty lines or
\item avoids $\smm{1& \\ &1}$ or $\smm{ &1\\1& }$.
\end{enumerate}
\end{thm}
\begin{proof}
Assume $P$ has four one-entries that do not share any row or column. Then those one-entries induce a $4\times4$ permutation inside $P$ and because $P$ does not contain any rotation of $P_1$, the induced permutation is either $1234$ or $4321$. Without loss of generality, assume is the the first case and denote the one-entries by $e_1,e_2,e_3$ and $e_4$.

For contradiction with the statement, assume $P$ also contains $P'=\smm{1& \\ &1}$. Clearly, any one-entry from $e_1,e_2,e_3$ and $e_4$ cannot be a part of any mapping of $P'$ because it would introduce a mapping of a rotation of $P_1$. Let $e_2=P[r_2,c_2]$ and $e_3=P[r_3,c_3]$. First, the submatrix $P[[r_2],[c_2,l]]$ does not contain $P'$ because then this copy of $P'$ together with $e_4$ would give us a rotated copy of $P_1$. Symmetrically, $P[[r_3,k],[c_3]]$ does not contain $P'$. Also, $P[[r_3-1],[c_3-1]]$ and $P[[r_2+1,k],[c_2+1,l]]$ are empty (else they would together with $e_2$ and $e_3$ give us a mapping of a rotation of $P_1$). Up to rotation, the only possible way to have $P'\im P$ is that the top one-entry of $P'$ is in the submatrix $P[[r_3-1],[c_2,c_3-1]]$ but then this entry together with $e_1$ and $e_3$ give us a mapping of a rotation of $P_1$ which is a contradiction.
\end{proof}
\begin{lemma}
Let $P\in\Pat$ be a pattern having only one non-empty line. Then for every inclusion maximal matrix $M\in\Mat$ avoiding $P$ the number of intervals of one-entries in a row is bounded by $O(k)$ and in column is bounded by $O(l)$. 
\end{lemma}
\begin{proof}
Let the non-empty line of $P$ be a row~$r$. Since $M$ is inclusion maximal, submatrices $M[[r-1],[n]]$ and $M[[m-r+1,m],[n]]$ contain no zero-entry, so each row contains just one interval of one-entries. If we look on any other row, it cannot contain $k$ one-entries, so the biggest possible number of intervals of one-entries is $k-1$.

Let us look on an arbitrary column~$c$ of $M$. If there is at least one one-entry in $M[[r,m-r],c]$ then because $M$ is inclusion maximal, the whole column is made of one-entries. Otherwise, there are two intervals of one-entries -- $M[[r-1],c]$ and $M[[m-r,m],c]$.
\end{proof}
\begin{lemma}
Let $P\in\Pat$ be a pattern having two non-empty lines. Then for every inclusion maximal matrix $M\in\Mat$ avoiding $P$ the number of intervals of one-entries in a row is bounded by $O(k^2+l)$ and in column is bounded by $O(k+l^2)$. 
\end{lemma}
\begin{proof}
First we assume the two non-empty lines of $P$ are rows $r_1<r_2$ (or symmetrically columns). From Observation~\ref{emptyrows} and inclusion maximality of $M$ we have that $M[[r_1-1],[n]]$ and $M[[m-r_2+1,m],[n]]$ contain no zero-entry. This way we restrict ourselves to case $r_1=1$ and $r_2=k$. From Theorem~\ref{smallintnum} we know that every inclusion maximal $N$ avoiding $P[\{r_1,r_2\},[n]]$ has at most $O(k^2)$ intervals of one-entries in each row and at most $1$ interval of one-entries in each column. From Theorem~\ref{emptymiddle} we also know that for given $M$ there is an inclusion maximal $N$ avoiding $P[\{r_1,r_2\},[n]]$ such that $M$ is submatrix of shifted and ORed copies of $N$. Since $M$ is inclusion maximal it is equal to those shifted and ORed copies of $N$ and since the number of intervals of $N$ is bounded, so is the number of intervals of $M$.

Now assume the two non-empty lines of $P$ are row~$r$ and column~$c$. We even prove that the number of intervals of one-entries is $O(k+l)$ in rows and $O(k+l)$ in columns. Also, because the proof is symmetric, we only show the bound for rows. The fact $M$ is inclusion maximal means that changing any zero-entry of $M$ to a one-entry creates a mapping of the forbidden pattern. We say that element~$e$ of $P$ is problematic, if it can be mapped to the changed one-entry of $M$. It can happen more elements are problematic for one zero-entry of $M$ but we assume there is always just one to get and upper bound of the number of intervals of ones. For simplicity of the proof, when we speak about two different zero-entries in a row, we mean two zero-entries separated by at least one one-entry. If the zero-entries are not separated by a one-entry, we consider them equivalent in terms of problematic elements.

Let us now fix an arbitrary row $M$ an talk about its zero-entries. For every one-entry~$e$ of the pattern except those in the $r$-th row, there is at most one zero-entry for which $e$ is problematic. For contradiction, assume there are two such zero-entries. Let the two created mappings with problematic $e$ after the change of those two zero-entries look like in Figure~\ref{twolines}, where one mapping is dashed and the other dotted. When we take the outer two vertical and horizontal lines, we have a mapping of $P$ that can use an existing one-entry to also map $e$ which gives us a contradiction with $M\in Av(P)$. Therefore, for each one-entry of $P$ besides those on the $r$-th row we have at most one interval of one-entries in $M$. For every one-entry~$e$ of $P$ from the $r$-th row and column~$c$ and every zero-entry of $M$ for which $e$ is problematic there can be as most $c-1$ one-entries to the left of it and at most $l-c$ one-entries to the right of it. If there are $k$ intervals of one-entries such that for zero-entries in between them problematic one-entries of $P$ are those from $r$-th row, then the zero-entry in the middle has either too many one-entries to the left or to the right.
\end{proof}
\begin{figure}[h!]
\centering
\includegraphics[height=60mm]{img/twolines.pdf}
\caption{Dashed and dotted lines resembling two different mappings of a forbidden pattern, where two horizontal lines show the boundaries of the mapping of row~$r$ and the vertical lines show boundaries of the mapping of column~$c$.}
\label{twolines}
\end{figure}
\begin{lemma}
Let $P\in\Pat$ be a pattern having three non-empty lines. Then for every inclusion maximal matrix $M\in\Mat$ avoiding $P$ the number of intervals of one-entries in a row is bounded by $O(f(k+l))$ and in column is bounded by $O(f(k+l))$. 
\end{lemma}
\begin{proof}
TODO - prove
\end{proof}
\begin{lemma}
Let $P\in\Pat$ be a pattern avoiding $\smm{0&1\\1&0}$ (or $I_2$). Then for every inclusion maximal matrix $M\in\Mat$ avoiding $P$ the number of intervals of one-entries in a row is bounded by $O(k+l)$ and in column is bounded by $O(k+l)$. 
\end{lemma}
\begin{proof}
From Theorem~\ref{walkingthm} we know that $P$ is a walking pattern. As such, it only contains up to $k+l-1$ one-entries. Assume $M$ has $k+l+1$ intervals of one-entries in some row~$r$. Then there are two zero entries in $r$-th row separated by at least one one-entry~$e'$ to which the same one-entry~$e=P[r_e,c_e]$ can be mapped when they are changed to one-entries. Call them $M[r_1,c_1]$ and $M[r_2,c_2]$. Clearly, $P[[r_e,k],[c_e]]$ besides $e$ can be mapped to $M[[r_1,m],[c_1]]$ and $P[[r_e],[c_e,l]]$ besides $e$ can be mapped to $M[[r_2],[c_2,n]]$. Also, $e$ can be mapped to $e'$ and since we mapped all one-entries of $P$ to $M$, we get a contradiction.
\end{proof}

Open questions:
\begin{itemize}
\item insertion of an empty column in between all columns of $P$
\end{itemize}

\section{Multiple patterns}
\begin{thm}
Let $P_1=\smm{0&0&1\\1&0&0}$ and $P_2=\smm{0&1\\0&0\\1&0}$, then for all $M$: $\PnimM\wedge\PnimM\Leftrightarrow M$ contains a walk $w$ and each one-entry $e$ is either on the walk $w$ or both element directly above $e$ and directly to the right of $e$ are on the walk $w$.
\begin{proof}
\begin{itemize}
\item[$\Rightarrow$] Let us take a walk $w$ containing all the left-most and bottom-most top-right empty elements of $M$. Clearly, every top-right ``corner'' entry of $w$ ($M[r,c]$ such that both $M[r+1,c]$ and $M[r,c-1]$ are on $w$) is a one-entry. Now consider for contradiction there is a one-entry anywhere but on $w$ or directly diagonally below any top-right corner of $w$. Then this one-entry together with at least one top-right corner of $w$ give us either $P_1$ or $P_2$ and thus a contradiction.
\item[$\Leftarrow$] If we take any one-entry~$e$, from the description of $M$ there is no one-entry that would create either of $P_1$ or $P_2$ with $e$.
\end{itemize}
\end{proof}
\end{thm}
\chapter*{Introduction}
\addcontentsline{toc}{chapter}{Introduction}
\textbf{TODO}:
\begin{itemize}
	\item Check all figures and their descriptions.
	\item Consider using more colors in figures.
	\item Fix or rewrite Lemma~\ref{lemma:maxmult}.
	\item Characterize or exclude $P_9$.
	\item Consider adding more patterns/generalizations.
	\item Maybe rewrite Definition 2.6.
	\item Consider proving Proposition 2.9 (currently commented).
	\item Consider rewriting Observation~\ref{obs:rel}.
	\item Find and check out Higman's Lemma (citing blindly now).
	\item Figure out what to do with Theorem 3.31.
	\item Fix or remove Lemma 3.29.
\end{itemize}

Throughout the paper, every time we speak about matrices we mean binary matrices (also called 01-matrices) and we omit the word binary. If we speak about a \emph{pattern}, we again mean a binary matrix and we use the word in order to distinguish among more matrices as well as to indicate relationship.

When dealing with matrices, we always index rows and column starting with one and when we speak about a row~$r$, we simply mean a row with index $r$. A \emph{line} is a common word for both a row and a column. When we order a set of lines, we first put all rows and then all columns. For $M\in\Mat$, $[m]$ is a set of all rows and $[m+n]$ is a set of all lines, where $m$-th element is the last row. This goes with the usual notation.
\begin{ntn}
For $n\in\mathbb{N}$ let $[n]:=\{1,2,\dots,n\}$ and for $m\in\mathbb{N}$, where $n\leq m$ let $[n,m]:=\{n,n+1,\dots,m\}$.
\end{ntn}
\begin{ntn}
For a matrix $M\in\{0,1\}^{m\times n}$ and $L\subseteq[m+n]$ let $M[L]$ denote a submatrix of $M$ induced by lines in $L$.
\end{ntn}
\begin{ntn}
For a matrix $M\in\Mat$, $R\subseteq[m]$ and $C\subseteq[n]$ let $M[R,C]$ denote a submatrix of $M$ induced by rows in $R$ and columns in $C$. Furthermore, for $r\in[m]$ and $c\in[n]$ let $M[r,c]:=M[\{r\},\{c\}]=M[\{r,c+m\}]$.
\end{ntn}
\begin{defn}
We say a matrix $M\in\Mat$ \emph{contains} a pattern $P\in\Pat$ \emph{as a submatrix} and denote it by $\PsmM$ if there are $R\in[m]$ and $C\in[n]$ such that $|R|=k$, $|C|=l$ and for every $r\in R$ and $c\in C$ if $P[r,c]=1$, then $M[R,C][r,c]=1$.
\end{defn}
This does not necessarily mean $P=M[R,C]$ as $M[R,C]$ can have more one-entries than $P$ does.
\begin{ntn}
For a matrix $M\in\Mat$ and $L\subseteq[m+n]$ let $M_{\preceq}[L]$ denote a matrix acquired from $M$ by applying following operation for each $l\in L$:
\begin{itemize}
\item If $l$ is the first row in $L$ then we replace the first $l$ rows by one row that is a bitwise OR of replaced rows.
\item If $l$ is the first column in $L$ then we replace the first $l-m$ columns by one column that is a bitwise OR of replaced columns.
\item Otherwise, we take $l$'s predecessor $l'\in L$ in the standard ordering and replace lines $[l'+1,l]$ by one line that is a bitwise OR of replaced lines.
\end{itemize}
\end{ntn}
\begin{ntn}
For a matrix $M\in\Mat$, $R\subseteq[m]$ and $C\subseteq[n]$ let $M_{\preceq}[R,C]:=M_{\preceq}[R\cup \{c+m|c\in C\}]$.
\end{ntn}
\begin{defn}
We say a matrix $M\in\Mat$ \emph{contains} a pattern $P\in\Pat$ \emph{as an interval minor} and denote it by $\PimM$ if there are $R\in[m]$ and $C\in[n]$ such that $|R|=k$, $|C|=l$ and for every $r\in R$ and $c\in C$ if $P[r,c]=1$, then $M_{\preceq}[R,C][r,c]=1$.
\end{defn}
%ordered bipartite graphs
%
%
%
%partitioning into rectangles (interval minors)
\begin{obs}
\label{obs1}
For all matrices $M$ and $P$, $\PsmM\Rightarrow\PimM$.
\end{obs}
\begin{obs}
For all matrices $M$ and $P$, if $P$ is a permutation matrix, then $\PsmM\Leftrightarrow\PimM$.
\end{obs}
\begin{proof}
If we have $\PimM$, then there is a partitioning of $M$ into rectangles and for each one-entry of $P$ there is at least one one-entry in the corresponding rectangle of $M$. Since $P$ is a permutation matrix, it is sufficient to take rows and columns having at least one one-entry in the right rectangle and we can always do so.

Together with Observation~\ref{obs1} this gives us the statement.
\end{proof}
\begin{obs}
Let $M\in\Mat$ and $P\in\Pat$, $\PimM\Leftrightarrow P^T\im M^T$.
\end{obs}
Because of this observation we will usually only show results only for rows or columns and expect both to hold and only show results for $P\in\Pat$ but assume the symmetrical results for $P^T$.

\begin{defn}
Let $\mathcal{F}$ be any class of forbidden matrices. We denote by $Av(\mathcal{F})$ the set of all matrices that avoid every $F\in\mathcal{F}$ as an interval minor.
\end{defn}
\begin{obs}
For all patterns $P,P'$: $P\im P'\Leftrightarrow\Avm{P}\subseteq\Avm{P'}$.
\end{obs}
\begin{proof}
Every $M\in\Avm{P}$ avoids $P$ and because $P\im P'$, it also avoids $P'$; therefore, it belongs to $\Avm{P'}$.

If $P\nim P'$ then $P'\in\Avm{P}$. As $P'\not\in\Avm{P'}$ we have $\Avm{P}\not\subseteq\Avm{P'}$.
\end{proof}

\section{Extremal function}
\begin{ntn}
Let $M$ be a matrix. We denote $|M|$ the weight of $M$, the number of one-entries in $M$.
\end{ntn}
Usually $|M|$ stands for a determinant of matrix $M$. However, in this paper we do not work with determinants at all so the notation should not lead to misunderstanding.
\begin{defn}
For a matrix $P$ we define $Ex(P,m,n):=\max\{|M||M\in\Mat,\ \PnsmM\}$. We denote $Ex(P,n):=Ex(P,n,n)$.
\end{defn}
\begin{defn}
For a matrix $P$ we define $Ex_{\preceq}(P,m,n):=max\{|M||M\in\Mat,\ \PnimM\}$. We denote $Ex_{\preceq}(P,n):=Ex_{\preceq}(P,n,n)$.
\end{defn}
\begin{obs}
For all $P,m,n$; $Ex_{\preceq}(P,m,n)\leq Ex(P,m,n)$.
\end{obs}
\begin{obs}
If $P\in\{0,1\}^{k\times l}$ has a one-entry at position $[a,b]$, then $$Ex(P,m,n)\geq\Big\{\begin{array}{ll}
m\cdot n & k>m\vee l>m \\
(k-1)n+(l-1)m-(k-1)(l-1) & \text{otherwise.}
\end{array}$$
\end{obs}
\begin{obs}
The same holds for $Ex_{\preceq}(P,m,n).$
\end{obs}
\begin{defn}
$P\in\{0,1\}^{k\times l}$ is \emph{(strongly) minimalist} if
$$Ex(P,m,n)=\Big\{\begin{array}{ll}
m\cdot n & k>m\vee l>m \\
(k-1)n+(l-1)m-(k-1)(l-1) & \text{otherwise.}
\end{array}$$
\end{defn}
\begin{defn}
$P\in\{0,1\}^{k\times l}$ is \emph{weakly minimalist} if
$$Ex_{\preceq}(P,m,n)=\Big\{\begin{array}{ll}
m\cdot n & k>m\vee l>m \\
(k-1)n+(l-1)m-(k-1)(l-1) & \text{otherwise.}
\end{array}$$
\end{defn}
\begin{obs}
If $P$ is strongly minimalist, then $P$ is weakly minimalist.
\end{obs}
\subsection{Known results}
\begin{fct}
\begin{enumerate}
\item $\smm{\bullet}$ is strongly minimalist.
\item If $P\in\Pat$ is strongly minimalist and there is a one-entry in the last row in the $c$-th column, then $P'\in\{0,1\}^{k+1\times l}$, which is created from $P$ by adding a new row having a one-entry only in the $c$-th column, is strongly minimalist.
\item If $P$ is strongly minimalist, then after changing a one-entry into a zero-entry it is still strongly minimalist.
\end{enumerate}
\end{fct}
% TODO: cite properly Interval minors of complete bipartite graphs - Bojan Mohar, Arash Rafiey
% Maybe mention that they were working with a different approach (ordered bigraphs) and also a different definition of an interval minor (allowing transposition) but it is not a problem for us and we show the proof in our notation.
\begin{fct}[\cite{bigraphs}]
Let $P=\{1\}^{2\times l}$, then $P$ is weakly minimalist.
\end{fct}
\begin{proof}
Let $M\in\Mat$ be a matrix avoiding $P=\{1\}^{2\times l}$ as an interval minor and $A_i=\{j\in[n]|\text{weight of }M[[i],\{j\}]>0\wedge \text{weight of }M[[i+1,m],\{j\}>0]\}$. Clearly $|A_i|\leq l-1$, otherwise $P\preceq M$. Let $b_j$ denote the number of one-entries in the $j$-th column. Each column~$j$ of $M$ appears in at least $b_j-1$ of sets $A_i$, $0\leq i\leq m-2$. It follows that
$$\text{weight of }M=\sum\limits_{j=0}^nb_j=\sum\limits_{j=0}^n(b_j-1)+n\leq\sum\limits_{i=0}^{m-2}|A_i|+n\leq(l-1)(m-1)+n$$
\end{proof}
This result is indeed very important because it shows that there are matrices like $\smm{11\\11}$, which are weakly minimalist, although it is known they are not strongly minimalist.
\begin{fct}[\cite{bigraphs}]
Let $P=\{1\}^{3\times l}$, then $P$ is weakly minimalist.
\end{fct}
\begin{proof}
Let $M\in\Mat$ be a matrix avoiding $P=\{1\}^{3\times l}$ as an interval minor and $A_i=\{j\in[n]|\text{ weight of }M[[i-1],\{j\}]>0\wedge \text{ weight of }M[[i+1,m],\{j\}>0\wedge M[i,j]\text{ one-entry}]\}$. Clearly $|A_i|\leq l-1$, otherwise $P\preceq M$. Let $b_j$ denote the number of one-entries in the $j$-th column. Each column~$j$ of $M$ (for which $b_j\geq2$) appears in exactly $b_j-2$ of sets $A_i$, $1\leq i\leq m-1$. It follows that
$$\text{weight of }M=\sum\limits_{j=0}^nb_j=\sum\limits_{j=0}^n(b_j-2)+2n\leq\sum\limits_{i=1}^{m-2}|A_i|+2n\leq(l-1)(m-2)+2n$$
\end{proof}